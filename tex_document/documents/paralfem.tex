% !Mode:: "TeX:UTF-8"
\documentclass{beamer}

\usetheme{Darmstadt}
\useinnertheme{rounded}

\usecolortheme{beaver}
%\usecolortheme{albatross}
%\usecolortheme{beetle}
%\usecolortheme{crane}
%\usecolortheme{dolphin}
%\usecolortheme{dove}
%\usecolortheme{fly}
%\usecolortheme{lily}
%\usecolortheme{orchid}
%\usecolortheme{rose}
%\usecolortheme{seagull}
%\usecolortheme{seahorse}
%\usecolortheme{whale}
%\usecolortheme{wolverine}
%\usecolortheme{default}


\setbeamerfont*{frametitle}{size=\normalsize,series=\bfseries}
\setbeamertemplate{navigation symbols}{}


%%%%%%%%------------------------------------------------------------------------
%%%% 日常所用宏包

%% 控制页边距
% 如果是beamer文档类, 则不用geometry
\makeatletter
\@ifclassloaded{beamer}{}{\usepackage[top=2.5cm, bottom=2.5cm, left=2.5cm, right=2.5cm]{geometry}}
\makeatother

\makeatletter
\@ifclassloaded{beamer}{
\makeatletter
\def\th@mystyle{%
    \normalfont % body font
    \setbeamercolor{block title example}{bg=orange,fg=white}
    \setbeamercolor{block body example}{bg=orange!20,fg=black}
    \def\inserttheoremblockenv{exampleblock}
  }
\makeatother
\theoremstyle{mystyle}
\newtheorem*{remark}{Remark}

\newcommand{\propnumber}{} % initialize
\newtheorem*{prop}{Proposition \propnumber}
\newenvironment{propc}[1]
  {\renewcommand{\propnumber}{#1}%
   \begin{shaded}\begin{prop}}
  {\end{prop}\end{shaded}}

\makeatletter
\newenvironment<>{proofs}[1][\proofname]{%
    \par
    \def\insertproofname{#1\@addpunct{.}}%
    \usebeamertemplate{proof begin}#2}
  {\usebeamertemplate{proof end}}
\makeatother

}{
}
\makeatother
\usepackage{amsthm}

%\DeclareMathOperator{\sech}{sech}
%\DeclareMathOperator{\csch}{csch}
%\DeclareMathOperator{\arcsec}{arcsec}
%\DeclareMathOperator{\arccot}{arccot}
%\DeclareMathOperator{\arccsc}{arccsc}
%\DeclareMathOperator{\arccosh}{arccosh}
%\DeclareMathOperator{\arcsinh}{arcsinh}
%\DeclareMathOperator{\arctanh}{arctanh}
%\DeclareMathOperator{\arcsech}{arcsech}
%\DeclareMathOperator{\arccsch}{arccsch}
%\DeclareMathOperator{\arccoth}{arccoth}
%% 控制项目列表
\usepackage{enumerate}

%% Todo list
\usepackage{enumitem}
\newlist{todolist}{itemize}{2}
\setlist[todolist]{label=$\square$}
\usepackage{pifont}
\newcommand{\cmark}{\ding{51}}%
\newcommand{\xmark}{\ding{55}}%
\newcommand{\done}{\rlap{$\square$}{\raisebox{2pt}{\large\hspace{1pt}\cmark}}%
\hspace{-2.5pt}}
\newcommand{\wontfix}{\rlap{$\square$}{\large\hspace{1pt}\xmark}}

\usepackage[utf8]{inputenc}
\usepackage[english]{babel}

\usepackage{framed}

%% 多栏显示
\usepackage{multicol}

%% 算法环境
\usepackage{algorithm}
\usepackage{algorithmic}
\usepackage{float}

%% 网址引用
\usepackage{url}

%% 控制矩阵行距
\renewcommand\arraystretch{1.4}

%% 粗体
\usepackage{lmodern}
\usepackage{bm}


%% hyperref宏包,生成可定位点击的超链接,并且会生成pdf书签
\makeatletter
\@ifclassloaded{beamer}{
\usepackage{hyperref}
\usepackage{ragged2e} % 对齐
}{
\usepackage[%
    pdfstartview=FitH,%
    CJKbookmarks=true,%
    bookmarks=true,%
    bookmarksnumbered=true,%
    bookmarksopen=true,%
    colorlinks=true,%
    citecolor=blue,%
    linkcolor=blue,%
    anchorcolor=green,%
    urlcolor=blue%
]{hyperref}
}
\makeatother



\makeatletter % 如果是 beamer 不需要下面两个包
\@ifclassloaded{beamer}{
\mode<presentation>
{
}
}{
%% 控制标题
\usepackage{titlesec}
%% 控制目录
\usepackage{titletoc}
}
\makeatother

%% 控制表格样式
\usepackage{booktabs}

%% 控制字体大小
\usepackage{type1cm}

%% 首行缩进,用\noindent取消某段缩进
\usepackage{indentfirst}

%% 支持彩色文本、底色、文本框等
\usepackage{color,xcolor}

%% AMS LaTeX宏包: http://zzg34b.w3.c361.com/package/maths.htm#amssymb
\usepackage{amsmath,amssymb}
%% 多个图形并排
\usepackage{subfig}
%%%% 基本插图方法
%% 图形宏包
\usepackage{graphicx}


%%%% 基本插图方法结束

%%%% pgf/tikz绘图宏包设置
\usepackage{pgf,tikz}
\usetikzlibrary{shapes,automata,snakes,backgrounds,arrows}
\usetikzlibrary{mindmap}
%% 可以直接在latex文档中使用graphviz/dot语言,
%% 也可以用dot2tex工具将dot文件转换成tex文件再include进来
%% \usepackage[shell,pgf,outputdir={docgraphs/}]{dot2texi}
%%%% pgf/tikz设置结束


\makeatletter % 如果是 beamer 不需要下面两个包
\@ifclassloaded{beamer}{

}{
%%%% fancyhdr设置页眉页脚
%% 页眉页脚宏包
\usepackage{fancyhdr}
%% 页眉页脚风格
\pagestyle{plain}
}

%% 有时会出现\headheight too small的warning
\setlength{\headheight}{15pt}

%% 清空当前页眉页脚的默认设置
%\fancyhf{}
%%%% fancyhdr设置结束


%% 设置listings宏包的一些全局样式
%% 参考http://hi.baidu.com/shawpinlee/blog/item/9ec431cbae28e41cbe09e6e4.html
\usepackage{listings}
\lstloadlanguages{[LaTeX]TeX}

\usepackage{fancyvrb}

\newenvironment{latexample}[1][language={[LaTeX]TeX}]
{\lstset{breaklines=true,
    prebreak = \raisebox{0ex}[0ex][0ex]{\ensuremath{\hookleftarrow}},
    frame=single,
    language={[LaTeX]TeX},
    showstringspaces=false,              %% 设定是否显示代码之间的空格符号
    numbers=left,                        %% 在左边显示行号
    numberstyle=\tiny,                   %% 设定行号字体的大小
    basicstyle=\scriptsize,                    %% 设定字体大小\tiny, \small, \Large等等
    keywordstyle=\color{blue!70}, commentstyle=\color{red!50!green!50!blue!50},
                                         %% 关键字高亮
    frame=shadowbox,                     %% 给代码加框
    rulesepcolor=\color{red!20!green!20!blue!20},
    escapechar=`,                        %% 中文逃逸字符,用于中英混排
    xleftmargin=2em,xrightmargin=2em, aboveskip=1em,
    %breaklines,                          %% 这条命令可以让LaTeX自动将长的代码行换行排版
    extendedchars=false                  %% 这一条命令可以解决代码跨页时,章节标题,页眉等汉字不显示的问题
    basicstyle=\footnotesize\ttfamily, #1}
  \VerbatimEnvironment\begin{VerbatimOut}{latexample.verb.out}}
  {\end{VerbatimOut}\noindent
  \begin{minipage}{1.05\linewidth}
    \lstinputlisting[]{latexample.verb.out}%
  \end{minipage}\qquad
  \begin{minipage}{1\linewidth}
    \input{latexample.verb.out}
  \end{minipage}\\
}

\usepackage{minted}
\renewcommand{\listingscaption}{Python code} \newminted{python}{
    escapeinside=||,
    mathescape=true,
    numbersep=5pt,
    linenos=true,
    autogobble,
    framesep=3mm}
%%%% listings宏包设置结束


%%%% 附录设置
\makeatletter % 对 beamer 要重新设置
\@ifclassloaded{beamer}{

}{
\usepackage[title,titletoc,header]{appendix}
}
\makeatother
%%%% 附录设置结束





%% 设定行距
\linespread{1}

%% 颜色
\newcommand{\red}{\color{red} }
\newcommand{\blue}{\color{blue} }
\newcommand{\brown}{\color{brown} }
\newcommand{\green}{\color{green} }

\newcommand{\bred}{\bf\color{red} }
\newcommand{\bblue}{\bf\color{blue} }
\newcommand{\bbrown}{\bf\color{brown} }
\newcommand{\bgreen}{\bf\color{green} }
%% 1. 小写的英文或希腊字母表示 标量或标量函数
%% 2. 大写的英文或希腊字母表示 集合或空间
%% 3. 粗体的小写字母代表向量或向量形式的常量和函数
%% 4. 粗体的大写字母代表矩阵或张量形式的常量和函数
%% 5. 空心大写字母代表特殊的空间 \mbR 实数 \mbC 复数 \mbP 多项式
%% 6. 花体的大写字母代表算子

%% 粗体的小写字母代表向量或向量函数
\newcommand{\bfa}{{\boldsymbol a}}
\newcommand{\bfb}{{\boldsymbol b}}
\newcommand{\bfc}{{\boldsymbol c}}
\newcommand{\bfd}{{\boldsymbol d}}
\newcommand{\bfe}{{\boldsymbol e}}
\newcommand{\bff}{{\boldsymbol f}}
\newcommand{\bfg}{{\boldsymbol g}}
\newcommand{\bfh}{{\boldsymbol h}}
\newcommand{\bfi}{{\boldsymbol i}}
\newcommand{\bfj}{{\boldsymbol j}}
\newcommand{\bfk}{{\boldsymbol k}}
\newcommand{\bfl}{{\boldsymbol l}}
\newcommand{\bfm}{{\boldsymbol m}}
\newcommand{\bfn}{{\boldsymbol n}}
\newcommand{\bfo}{{\boldsymbol o}}
\newcommand{\bfp}{{\boldsymbol p}}
\newcommand{\bfq}{{\boldsymbol q}}
\newcommand{\bfr}{{\boldsymbol r}}
\newcommand{\bfs}{{\boldsymbol s}}
\newcommand{\bft}{{\boldsymbol t}}
\newcommand{\bfu}{{\boldsymbol u}}
\newcommand{\bfv}{{\boldsymbol v}}
\newcommand{\bfw}{{\boldsymbol w}}
\newcommand{\bfx}{{\boldsymbol x}}
\newcommand{\bfy}{{\boldsymbol y}}
\newcommand{\bfz}{{\boldsymbol z}}

%  算子
\newcommand{\mca}{{\mathcal a}}
\newcommand{\mcb}{{\mathcal b}}
\newcommand{\mcc}{{\mathcal c}}
\newcommand{\mcd}{{\mathcal d}}
\newcommand{\mce}{{\mathcal e}}
\newcommand{\mcf}{{\mathcal f}}
\newcommand{\mcg}{{\mathcal g}}
\newcommand{\mch}{{\mathcal h}}
\newcommand{\mci}{{\mathcal i}}
\newcommand{\mcj}{{\mathcal j}}
\newcommand{\mck}{{\mathcal k}}
\newcommand{\mcl}{{\mathcal l}}
\newcommand{\mcm}{{\mathcal m}}
\newcommand{\mcn}{{\mathcal n}}
\newcommand{\mco}{{\mathcal o}}
\newcommand{\mcp}{{\mathcal p}}
\newcommand{\mcq}{{\mathcal q}}
\newcommand{\mcr}{{\mathcal r}}
\newcommand{\mcs}{{\mathcal s}}
\newcommand{\mct}{{\mathcal t}}
\newcommand{\mcu}{{\mathcal u}}
\newcommand{\mcv}{{\mathcal v}}
\newcommand{\mcw}{{\mathcal w}}
\newcommand{\mcx}{{\mathcal x}}
\newcommand{\mcy}{{\mathcal y}}
\newcommand{\mcz}{{\mathcal z}}

% \rmd
\newcommand{\mra}{{\mathrm a}}
\newcommand{\mrb}{{\mathrm b}}
\newcommand{\mrc}{{\mathrm c}}
\newcommand{\mrd}{{\mathrm d}}
\newcommand{\mre}{{\mathrm e}}
\newcommand{\mrf}{{\mathrm f}}
\newcommand{\mrg}{{\mathrm g}}
\newcommand{\mrh}{{\mathrm h}}
\newcommand{\mri}{{\mathrm i}}
\newcommand{\mrj}{{\mathrm j}}
\newcommand{\mrk}{{\mathrm k}}
\newcommand{\mrl}{{\mathrm l}}
\newcommand{\mrm}{{\mathrm m}}
\newcommand{\mrn}{{\mathrm n}}
\newcommand{\mro}{{\mathrm o}}
\newcommand{\mrp}{{\mathrm p}}
\newcommand{\mrq}{{\mathrm q}}
\newcommand{\mrr}{{\mathrm r}}
\newcommand{\mrs}{{\mathrm s}}
\newcommand{\mrt}{{\mathrm t}}
\newcommand{\mru}{{\mathrm u}}
\newcommand{\mrv}{{\mathrm v}}
\newcommand{\mrw}{{\mathrm w}}
\newcommand{\mrx}{{\mathrm x}}
\newcommand{\mry}{{\mathrm y}}
\newcommand{\mrz}{{\mathrm z}}

%% 粗体的大写字母一般表示矩阵和张量
\newcommand{\bfA}{{\boldsymbol A}}
\newcommand{\bfB}{{\boldsymbol B}}
\newcommand{\bfC}{{\boldsymbol C}}
\newcommand{\bfD}{{\boldsymbol D}}
\newcommand{\bfE}{{\boldsymbol E}}
\newcommand{\bfF}{{\boldsymbol F}}
\newcommand{\bfG}{{\boldsymbol G}}
\newcommand{\bfH}{{\boldsymbol H}}
\newcommand{\bfI}{{\boldsymbol I}}
\newcommand{\bfJ}{{\boldsymbol J}}
\newcommand{\bfK}{{\boldsymbol K}}
\newcommand{\bfL}{{\boldsymbol L}}
\newcommand{\bfM}{{\boldsymbol M}}
\newcommand{\bfN}{{\boldsymbol N}}
\newcommand{\bfO}{{\boldsymbol O}}
\newcommand{\bfP}{{\boldsymbol P}}
\newcommand{\bfQ}{{\boldsymbol Q}}
\newcommand{\bfR}{{\boldsymbol R}}
\newcommand{\bfS}{{\boldsymbol S}}
\newcommand{\bfT}{{\boldsymbol T}}
\newcommand{\bfU}{{\boldsymbol U}}
\newcommand{\bfV}{{\boldsymbol V}}
\newcommand{\bfW}{{\boldsymbol W}}
\newcommand{\bfX}{{\boldsymbol X}}
\newcommand{\bfY}{{\boldsymbol Y}}
\newcommand{\bfZ}{{\boldsymbol Z}}

%% 花体大写字母
\newcommand{\mcA}{{\mathcal A}}
\newcommand{\mcB}{{\mathcal B}}
\newcommand{\mcC}{{\mathcal C}}
\newcommand{\mcD}{{\mathcal D}}
\newcommand{\mcE}{{\mathcal E}}
\newcommand{\mcF}{{\mathcal F}}
\newcommand{\mcG}{{\mathcal G}}
\newcommand{\mcH}{{\mathcal H}}
\newcommand{\mcI}{{\mathcal I}}
\newcommand{\mcJ}{{\mathcal J}}
\newcommand{\mcK}{{\mathcal K}}
\newcommand{\mcL}{{\mathcal L}}
\newcommand{\mcM}{{\mathcal M}}
\newcommand{\mcN}{{\mathcal N}}
\newcommand{\mcO}{{\mathcal O}}
\newcommand{\mcP}{{\mathcal P}}
\newcommand{\mcQ}{{\mathcal Q}}
\newcommand{\mcR}{{\mathcal R}}
\newcommand{\mcS}{{\mathcal S}}
\newcommand{\mcT}{{\mathcal T}}
\newcommand{\mcU}{{\mathcal U}}
\newcommand{\mcV}{{\mathcal V}}
\newcommand{\mcW}{{\mathcal W}}
\newcommand{\mcX}{{\mathcal X}}
\newcommand{\mcY}{{\mathcal Y}}
\newcommand{\mcZ}{{\mathcal Z}}

%% 空心大写字母
\newcommand{\mbA}{{\mathbb A}}
\newcommand{\mbB}{{\mathbb B}}
\newcommand{\mbC}{{\mathbb C}}
\newcommand{\mbD}{{\mathbb D}}
\newcommand{\mbE}{{\mathbb E}}
\newcommand{\mbF}{{\mathbb F}}
\newcommand{\mbG}{{\mathbb G}}
\newcommand{\mbH}{{\mathbb H}}
\newcommand{\mbI}{{\mathbb I}}
\newcommand{\mbJ}{{\mathbb J}}
\newcommand{\mbK}{{\mathbb K}}
\newcommand{\mbL}{{\mathbb L}}
\newcommand{\mbM}{{\mathbb M}}
\newcommand{\mbN}{{\mathbb N}}
\newcommand{\mbO}{{\mathbb O}}
\newcommand{\mbP}{{\mathbb P}}
\newcommand{\mbQ}{{\mathbb Q}}
\newcommand{\mbR}{{\mathbb R}}
\newcommand{\mbS}{{\mathbb S}}
\newcommand{\mbT}{{\mathbb T}}
\newcommand{\mbU}{{\mathbb U}}
\newcommand{\mbV}{{\mathbb V}}
\newcommand{\mbW}{{\mathbb W}}
\newcommand{\mbX}{{\mathbb X}}
\newcommand{\mbY}{{\mathbb Y}}
\newcommand{\mbZ}{{\mathbb Z}}

\newcommand{\mrA}{{\mathrm A}}
\newcommand{\mrB}{{\mathrm B}}
\newcommand{\mrC}{{\mathrm C}}
\newcommand{\mrD}{{\mathrm D}}
\newcommand{\mrE}{{\mathrm E}}
\newcommand{\mrF}{{\mathrm F}}
\newcommand{\mrG}{{\mathrm G}}
\newcommand{\mrH}{{\mathrm H}}
\newcommand{\mrI}{{\mathrm I}}
\newcommand{\mrJ}{{\mathrm J}}
\newcommand{\mrK}{{\mathrm K}}
\newcommand{\mrL}{{\mathrm L}}
\newcommand{\mrM}{{\mathrm M}}
\newcommand{\mrN}{{\mathrm N}}
\newcommand{\mrO}{{\mathrm O}}
\newcommand{\mrP}{{\mathrm P}}
\newcommand{\mrQ}{{\mathrm Q}}
\newcommand{\mrR}{{\mathrm R}}
\newcommand{\mrS}{{\mathrm S}}
\newcommand{\mrT}{{\mathrm T}}
\newcommand{\mrU}{{\mathrm U}}
\newcommand{\mrV}{{\mathrm V}}
\newcommand{\mrW}{{\mathrm W}}
\newcommand{\mrX}{{\mathrm X}}
\newcommand{\mrY}{{\mathrm Y}}
\newcommand{\mrZ}{{\mathrm Z}}


% 粗体的 Greek 字母
\newcommand{\balpha}{{\bm \alpha}}
\newcommand{\bbeta}{{\bm \beta}}
\newcommand{\bgamma}{{\bm \gamma}}
\newcommand{\bdelta}{{\bm \delta}}
\newcommand{\bepsilon}{{\bm \epsilon}}
\newcommand{\bvarepsilon}{{\bm \varepsilon}}
\newcommand{\bzeta}{{\bm \zeta}}
\newcommand{\bfeta}{{\bm \eta}}
\newcommand{\btheta}{{\bm \theta}}
\newcommand{\biota}{{\bm \iota}}
\newcommand{\bkappa}{{\bm \kappa}}
\newcommand{\blambda}{{\bm \lambda}}
\newcommand{\bmu}{{\bm \mu}}
\newcommand{\bnu}{{\bm \nu}}
\newcommand{\bxi}{{\bm \xi}}
\newcommand{\bomicron}{{\bm \omicron}}
\newcommand{\bpi}{{\bm \pi}}
\newcommand{\brho}{{\bm \rho}}
\newcommand{\bsigma}{{\bm \sigma}}
\newcommand{\btau}{{\bm \tau}}
\newcommand{\bupsilon}{{\bm \upsilon}}
\newcommand{\bphi}{{\bm \phi}}
\newcommand{\bvarphi}{{\bm \varphi}}
\newcommand{\bchi}{{\bm \chi}}
\newcommand{\bpsi}{{\bm \psi}}

\newcommand{\bAlpha}{{\bm \Alpha}}
\newcommand{\bBeta}{{\bm \Beta}}
\newcommand{\bGamma}{{\bm \Gamma}}
\newcommand{\bDelta}{{\bm \Delta}}
\newcommand{\bEpsilon}{{\bm \Psilon}}
\newcommand{\bVarepsilon}{{\bm \Varepsilon}}
\newcommand{\bZeta}{{\bm \Zeta}}
\newcommand{\bEta}{{\bm \Eta}}
\newcommand{\bTheta}{{\bm \Theta}}
\newcommand{\bIota}{{\bm \Iota}}
\newcommand{\bKappa}{{\bm \Kappa}}
\newcommand{\bLambda}{{\bm \Lambda}}
\newcommand{\bMu}{{\bm \Mu}}
\newcommand{\bNu}{{\bm \Nu}}
\newcommand{\bXi}{{\bm \Xi}}
\newcommand{\bOmicron}{{\bm \Omicron}}
\newcommand{\bPi}{{\bm \Pi}}
\newcommand{\bRho}{{\bm \Rho}}
\newcommand{\bSigma}{{\bm \Sigma}}
\newcommand{\bTau}{{\bm \Tau}}
\newcommand{\bUpsilon}{{\bm \Upsilon}}
\newcommand{\bPhi}{{\bm \Phi}}
\newcommand{\bChi}{{\bm \Chi}}
\newcommand{\bPsi}{{\bm \Psi}}

% \int_\Omega \bfx^2 \rmd \bfx
\newcommand{\rmd}{\,\mathrm d}
\newcommand{\bfzero}{\mathbf 0}

%% 算子
\newcommand{\ospan}{\operatorname{span}}
\newcommand{\odiv}{\operatorname{div}}
\newcommand{\otr}{\operatorname{tr}}
\newcommand{\ograd}{\operatorname{grad}}
\newcommand{\orot}{\operatorname{rot}}
\newcommand{\ocurl}{\operatorname{curl}}
\newcommand{\odist}{\operatorname{dist}}
\newcommand{\osign}{\operatorname{sign}}
\newcommand{\odiag}{\operatorname{diag}}
\newcommand{\oran}{\operatorname{Ran}} % 像空间
\newcommand{\oker}{\operatorname{Ker}} % 核空间
\newcommand{\ore}{\operatorname{Re}} % 实部
\newcommand{\oim}{\operatorname{Im}} % 虚部
\newcommand{\orank}{\operatorname{rank}}
\newcommand{\ovec}{\operatorname{vec}}
\newcommand{\odet}{\operatorname{det}}
\newcommand{\odim}{\operatorname{dim}}
\newcommand{\osym}{\operatorname{sym}}

\newcommand{\obcurl}{\operatorname{\bf curl}}
%%%% 个性设置结束
%%%%%%%%------------------------------------------------------------------------


%%%%%%%%------------------------------------------------------------------------
%%%% bibtex设置

%% 设定参考文献显示风格
% 下面是几种常见的样式
% * plain: 按字母的顺序排列,比较次序为作者、年度和标题
% * unsrt: 样式同plain,只是按照引用的先后排序
% * alpha: 用作者名首字母+年份后两位作标号,以字母顺序排序
% * abbrv: 类似plain,将月份全拼改为缩写,更显紧凑
% * apalike: 美国心理学学会期刊样式, 引用样式 [Tailper and Zang, 2006]

%\makeatletter
%\@ifclassloaded{beamer}{
%\bibliographystyle{apalike}
%}{
%\bibliographystyle{abbrv}
%}
%\makeatother


%%%% bibtex设置结束
%%%%%%%%------------------------------------------------------------------------

%%%%%%%%------------------------------------------------------------------------
%%%% xeCJK相关宏包

\usepackage{xltxtra, fontenc, xunicode}
\usepackage[slantfont, boldfont]{xeCJK}

\setlength{\parindent}{1.5em}%中文缩进两个汉字位

%% 针对中文进行断行
\XeTeXlinebreaklocale "zh"

%% 给予TeX断行一定自由度
\XeTeXlinebreakskip = 0pt plus 1pt minus 0.1pt

%%%% xeCJK设置结束
%%%%%%%%------------------------------------------------------------------------

%%%%%%%%------------------------------------------------------------------------
%%%% xeCJK字体设置

%% 设置中文标点样式,支持quanjiao、banjiao、kaiming等多种方式
\punctstyle{kaiming}

%% 设置缺省中文字体
\setCJKmainfont[BoldFont={Adobe Heiti Std}, ItalicFont={Adobe Kaiti Std}]{Adobe Song Std}
%\setCJKmainfont{Adobe Kaiti Std}
%% 设置中文无衬线字体
%\setCJKsansfont[BoldFont={Adobe Heiti Std}]{Adobe Kaiti Std}
%% 设置等宽字体
%\setCJKmonofont{Adobe Heiti Std}

%% 英文衬线字体
\setmainfont{DejaVu Serif}
%% 英文等宽字体
\setmonofont{DejaVu Sans Mono}
%% 英文无衬线字体
\setsansfont{DejaVu Sans}

%% 定义新字体
\setCJKfamilyfont{song}{Adobe Song Std}
\setCJKfamilyfont{kai}{Adobe Kaiti Std}
\setCJKfamilyfont{hei}{Adobe Heiti Std}
\setCJKfamilyfont{fangsong}{Adobe Fangsong Std}
\setCJKfamilyfont{lisu}{LiSu}
\setCJKfamilyfont{youyuan}{YouYuan}

%% 自定义宋体
\newcommand{\song}{\CJKfamily{song}}
%% 自定义楷体
\newcommand{\kai}{\CJKfamily{kai}}
%% 自定义黑体
\newcommand{\hei}{\CJKfamily{hei}}
%% 自定义仿宋体
\newcommand{\fangsong}{\CJKfamily{fangsong}}
%% 自定义隶书
\newcommand{\lisu}{\CJKfamily{lisu}}
%% 自定义幼圆
\newcommand{\youyuan}{\CJKfamily{youyuan}}

%%%% xeCJK字体设置结束
%%%%%%%%------------------------------------------------------------------------

%%%%%%%%------------------------------------------------------------------------
%%%% 一些关于中文文档的重定义
\newcommand{\chntoday}{\number\year\,年\,\number\month\,月\,\number\day\,日}
%% 数学公式定理的重定义

%% 中文破折号,据说来自清华模板
\newcommand{\pozhehao}{\kern0.3ex\rule[0.8ex]{2em}{0.1ex}\kern0.3ex}

\makeatletter %
\@ifclassloaded{beamer}{

}{
\newtheorem{example}{例}
\newtheorem{theorem}{定理}[section]
\newtheorem{definition}{定义}
\newtheorem{axiom}{公理}
\newtheorem{property}{性质}
\newtheorem{proposition}{命题}
\newtheorem{lemma}{引理}
\newtheorem{corollary}{推论}
\newtheorem{remark}{注解}
\newtheorem{condition}{条件}
\newtheorem{conclusion}{结论}
\newtheorem{assumption}{假设}
}
\makeatother

\makeatletter %
\@ifclassloaded{beamer}{

}{
%% 章节等名称重定义
\renewcommand{\contentsname}{目录}
\renewcommand{\indexname}{索引}
\renewcommand{\listfigurename}{插图目录}
\renewcommand{\listtablename}{表格目录}
\renewcommand{\appendixname}{附录}
\renewcommand{\appendixpagename}{附录}
\renewcommand{\appendixtocname}{附录}
\@ifclassloaded{book}{

}{
\renewcommand{\abstractname}{摘要}
}
}
\makeatother

\renewcommand{\figurename}{图}
\renewcommand{\tablename}{表}

\makeatletter
\@ifclassloaded{book}{
\renewcommand{\bibname}{参考文献}
}{
\renewcommand{\refname}{参考文献}
}
\makeatother

\floatname{algorithm}{算法}
\renewcommand{\algorithmicrequire}{\textbf{输入:}}
\renewcommand{\algorithmicensure}{\textbf{输出:}}

\renewcommand{\today}{\number\year 年 \number\month 月 \number\day 日}

%%%% 中文重定义结束
%%%%%%%%------------------------------------------------------------------------


\usepackage{biblatex}
\addbibresource{ref.bib}

\usefonttheme[onlymath]{serif}
\numberwithin{subsection}{section}
%\usefonttheme[onlylarge]{structurebold}
\setbeamercovered{transparent}

\title{基于 FEALPy 的拓扑优化中的自动微分框架} 
\author{何亮}
\institute[XTU]{
学号:202331510117\\
\vspace{5pt}
湘潭大学$\bullet$数学与计算科学学院\\
}
 
\date[XTU]
{
    \today
}


\AtBeginSection[]
{
  \frame<beamer>{ 
    \frametitle{Outline}   
    \tableofcontents[currentsection] 
  }
}

\AtBeginSubsection[]
{
  \frame<beamer>{ 
    \frametitle{Outline}   
    \tableofcontents[currentsubsection] 
  }
}

\begin{document}
\begin{frame}
  \titlepage
\end{frame}

\begin{frame}{Outline}
  \tableofcontents
\end{frame}

\section{引言}

%--------------------------------------------------------------------------------------
\begin{frame}
\frametitle{研究背景}
    \begin{itemize}
	    \item[•]拓扑优化(TO)是一种通过优化材料分布以获得最佳结构性能的技术,在工程设计中具有重要应用价值。
	    \vspace{0.3cm}
	    \item[•]自动微分(AD)技术可以高效、准确地计算数值函数的导数,有潜力在拓扑优化中用于灵敏度分析。
    \end{itemize}
\end{frame}
%--------------------------------------------------------------------------------------

%--------------------------------------------------------------------------------------
\begin{frame}
\frametitle{研究动机}
    \begin{itemize}
	    \item[•]尽管 AD 技术在机器学习和科学计算中广泛应用,但在拓扑优化中的应用仍较少,尤其是将其与传统方法进行对比的研究更少。
	    \vspace{0.3cm}
	    \item[•]现有的自动微分 TO 框架展示了 AD 在拓扑优化中的应用潜力,但其可重用性和扩展性存在局限,需要进一步改进。
    \end{itemize}
\end{frame}
%--------------------------------------------------------------------------------------

%--------------------------------------------------------------------------------------
\begin{frame}
\frametitle{研究目标}
    \begin{itemize}
	    \item[•]本研究旨在基于 FEALPy 框架,应用 AD 技术提高拓扑优化的灵敏度分析效率和准确性。
	    \vspace{0.3cm}
	    \item[•]实现重用性更好的自动微分拓扑优化框架,并将其与传统变密度方法进行对比,验证 AD 对变密度方法的影响。
    \end{itemize}
\end{frame}
%--------------------------------------------------------------------------------------

\section{背景与动机}

%--------------------------------------------------------------------------------------
\begin{frame}
    \frametitle{背景}
    \textbf{TO 已有工作}
    \begin{itemize}
	    \item[$\bullet$] TO 在近年来得到了广泛关注,主要由于制造能力和计算建模的进步。
	    \vspace{0.3cm}
	    \item[$\bullet$] 基于连续体结构的 TO 方法目前分为两大类:变密度方法及其变形(如 SIMP 方法、ESO/BESO 方法)、边界演化方法(如水平集方法、MMC/MMV 方法和相场方法)。
	    \vspace{0.3cm}
	    \item[$\bullet$] 这些方法各有优缺点,但在处理大规模问题时通常面临计算效率和灵敏度分析复杂的问题。
    \end{itemize}
\end{frame}
%--------------------------------------------------------------------------------------

%--------------------------------------------------------------------------------------
\begin{frame}
	\frametitle{背景}
    \textbf{AD 已有工作}
    \begin{itemize}
	    \item[$\bullet$] AD 是一种强大的技术,通过在计算过程中自动生成导数,避免了手动求导的繁琐和易错。
	    \vspace{0.3cm}
	    \item[$\bullet$] AD 已经存在了几十年,被广泛应用于解决各种问题,从分子动力学模拟、流动设计问题到光子晶体的设计等,提高了计算效率和准确性。
    \end{itemize}
\end{frame}
%--------------------------------------------------------------------------------------

%--------------------------------------------------------------------------------------
\begin{frame}

	\frametitle{背景}
    \textbf{AuTO 已有工作}
    \begin{itemize}
	    \item[$\bullet$] AuTO(Automatic differentiation in Topology Optimization)框架结合了拓扑优化和自动微分技术,实现了高效的灵敏度分析。
	    \vspace{0.3cm}
	    \item[$\bullet$] Aaditya Chandrasekhar 基于高性能 Python 库 JAX,自动计算了用户定义的 TO 问题的灵敏度。
	    \vspace{0.3cm}
	    \item[$\bullet$] AuTO 框架展示了在复杂拓扑优化问题中自动微分的应用潜力,但在可重用性和扩展性方面存在局限。
    \end{itemize}
    
\end{frame}

%--------------------------------------------------------------------------------------

\begin{frame}
    \frametitle{动机}    
    \textbf{基于 FEALPy 的原因}
    \begin{itemize}
        \item[$\bullet$]\textbf{模块化设计}:FEALPy 的模块化设计提供了灵活的数值计算环境,便于算法的开发和测试。
        \vspace{0.3cm}
        \item[$\bullet$]\textbf{丰富的网格生成算法}:FEALPy 中实现了大量的网格生成和优化算法,能够处理复杂的几何结构。
        \vspace{0.3cm}
        \item[$\bullet$]\textbf{Torch 模块与自动微分}:FEALPy 中的 Torch 模块借助 PyTorch 实现 AD,提高了灵敏度分析的效率和准确性。
    \end{itemize}
\end{frame}

%--------------------------------------------------------------------------------------

\begin{frame}
    \frametitle{动机}
    \textbf{研究价值}
    \begin{itemize}
        \item[$\bullet$]\textbf{多计算内核支持}:FEALPy 作为一个开源的软件包,可以使用不同的计算内核(如 JAX、PyTorch)来实现 AuTO。
        \vspace{0.3cm}
        \item[$\bullet$]\textbf{结果对比验证}:FEALPy 基于 Numpy 也实现了很多传统 TO 方法,可以很方便地将两者的结果进行对比验证。
        \vspace{0.3cm}
        \item[$\bullet$]\textbf{探索 AD 技术应用}:探讨 AD 技术在复杂非线性和多物理场问题中的应用,填补现有研究的空白。
    \end{itemize}
    
\end{frame}

%--------------------------------------------------------------------------------------

\section{研究目标}

%--------------------------------------------------------------------------------------
\begin{frame}
    \frametitle{研究目标}
    \begin{block}{总体目标}
		本研究旨在 FEALPy 框架下,应用 AD 技术提高拓扑优化的灵敏度分析效率和准确性,并实现重用性更好的 AuTO 框架。
	\end{block}
\end{frame}
%--------------------------------------------------------------------------------------

%--------------------------------------------------------------------------------------
\begin{frame}
	\frametitle{研究目标}
	\textbf{具体目标}
    \begin{itemize}
        \item[$\bullet$]\textbf{利用 FEALPy 的模块化设计}
        \begin{itemize}
            \item[•]通过模块化设计,实现更加灵活和高效的拓扑优化算法流程,便于拓扑优化算法的开发、测试和扩展。
        \end{itemize}
        \vspace{0.3cm}
        \item[$\bullet$]\textbf{利用 FEALPy 的 Torch 模块}
        \begin{itemize}
	        \item[•]实现基于 Torch 的灵敏度分析模块,通过 PyTorch 实现高效的自动微分,简化求导过程,优化拓扑优化流程。
        \end{itemize}
        \vspace{0.3cm}
        \item[$\bullet$]\textbf{探讨 AD 技术在复杂非线性和多物理场问题中的应用}
        \begin{itemize}
            \item[•]扩展 AD 技术在复杂非线性问题中的应用,提高计算稳定性和精度。
            \item[•]探索 AD 技术在多物理场耦合问题中的应用,解决多物理场问题中的灵敏度分析难题。
        \end{itemize}
    \end{itemize}
\end{frame}
%--------------------------------------------------------------------------------------

\section{研究方法}

%--------------------------------------------------------------------------------------
\begin{frame}
	\frametitle{研究方法}
	 \textbf{基于变密度方法的柔顺度最小化问题数学模型}
	 \begin{itemize}
	            \item 使用张量分析语言描述变密度方法的柔顺度最小化问题。
	            \item[•]目标函数
	            $$\min_{\rho} :C = \int_{\Omega}(E(\rho)\varepsilon(u):\varepsilon(u))~\mathrm{d}x.$$
	            \item[•]约束条件
	            $$\int_\Omega\rho(x)~\mathrm{d}x - V_{\max};\quad0<\rho_{\min}\leq\rho\leq1.$$
	            \item[•]平衡方程
	            $$\int_{\Omega}(\sigma(u):\varepsilon(v))~\mathrm{d}x = \int_{\Omega}f\cdot{v}~\mathrm{d}x + \int_{\Gamma_N}F_n\cdot{v}~\mathrm{d}s,\quad\mathrm{for~all}~v\in{U}.$$
	 \end{itemize}
\end{frame}
%--------------------------------------------------------------------------------------

%--------------------------------------------------------------------------------------
\begin{frame}
	\frametitle{研究方法}
	 \textbf{基于 FEALPy 中的 Numpy 模块实现传统变密度方法的算法流程}
	 \begin{itemize}
            \item[•]初始化设计变量 $\rho$。
            \item[•]迭代过程:
	        \begin{enumerate}
	            \item[1.]计算当前设计变量下的刚度矩阵 $K(\rho)$。
	            \item[2.]求解平衡方程 $K(\rho) u = f$ 获得位移场 $u$。
	            \item[3.]计算目标函数 $C(\rho, u)$。
	            \item[4.]进行灵敏度分析,计算目标函数对设计变量的导数。
	            $\frac{\partial C}{\partial \rho}$。
	            \item[5.]更新设计变量 $\rho$。
	        \end{enumerate}
	        \item[•]收敛判定:检查目标函数变化是否满足精度要求或达到最大迭代次数。
     \end{itemize}
\end{frame}
%--------------------------------------------------------------------------------------

%--------------------------------------------------------------------------------------
\begin{frame}
	\frametitle{研究方法}
	 \textbf{基于 FEALPy 的 Torch 模块实现引入 AD 的变密度方法的算法流程}
	 \begin{itemize}
            \item[•]初始化设计变量 $\rho$。
            \item[•]迭代过程:
	        \begin{enumerate}
	            \item[1.]计算当前设计变量下的刚度矩阵 $K(\rho)$。
	            \item[2.]求解平衡方程 $K(\rho) u = f$ 获得位移场 $u$。
	            \item[3.]计算目标函数 $C(\rho, u)$。
	            \item[4.]使用 PyTorch 自动计算目标函数对设计变量的导数。 
	            $\frac{\partial C}{\partial \rho}$。
	            \item[5.]更新设计变量 $\rho$。
	        \end{enumerate}
	        \item[•]收敛判定:检查目标函数变化是否满足精度要求或达到最大迭代次数。
     \end{itemize}
\end{frame}
%--------------------------------------------------------------------------------------

%--------------------------------------------------------------------------------------
\begin{frame}
	\frametitle{研究方法}
	 \textbf{比较两者的结果,验证 AD 对于变密度方法的影响}
    \begin{itemize}
        \item[•]比较两种方法在计算时间、收敛速度和计算精度上的表现。
        \vspace{0.3cm}
        \item[•]分析自动微分在复杂问题中的优势和可能的劣势。
        \vspace{0.3cm}
        \item[•]通过数值实验验证自动微分技术在变密度方法中的应用效果。
    \end{itemize}
\end{frame}
%--------------------------------------------------------------------------------------

\section{研究进展}

%--------------------------------------------------------------------------------------
\begin{frame}
    \frametitle{研究进展}
    \textbf{基于变密度方法的柔顺度最小化问题数学模型的推导}
    \vspace{0.3cm}
    \begin{itemize}
        \item[•]完成工作:推导了基于变密度方法的柔顺度最小化问题的数学模型。
    \end{itemize}
\end{frame}
%--------------------------------------------------------------------------------------

%--------------------------------------------------------------------------------------
\begin{frame}
    \frametitle{研究进展}
    \textbf{基于 FEALPy 中的 Numpy 模块实现传统变密度方法}
    \vspace{0.3cm}
    \begin{itemize}
        \item[•]计划工作:基于 FEALPy 中的 Numpy 模块,按照推导的数学模型实现传统变密度方法的求解流程。
    \end{itemize}
\end{frame}
%--------------------------------------------------------------------------------------

%--------------------------------------------------------------------------------------
\begin{frame}
    \frametitle{研究进展}
    \textbf{基于 FEALPy 的 Torch 模块实现引入 AD 的变密度方法}
    \vspace{0.3cm}
    \begin{itemize}
        \item[•]计划工作:基于 FEALPy 的 Torch 模块,利用 PyTorch 的自动微分功能实现引入 AD 的变密度方法的求解流程。
    \end{itemize}
\end{frame}
%--------------------------------------------------------------------------------------

%--------------------------------------------------------------------------------------
\begin{frame}
    \frametitle{研究进展}
    \textbf{比较两者的结果,验证 AD 对于变密度方法的影响}
    \vspace{0.3cm}
    \begin{itemize}
        \item[•]计划工作:完成基于传统变密度方法和引入 AD 的变密度方法的实现后,比较两者在计算时间、收敛速度和计算精度上的表现,验证自动微分技术的应用效果。
    \end{itemize}
\end{frame}
%--------------------------------------------------------------------------------------

\section{预期结果与贡献}

%--------------------------------------------------------------------------------------
\begin{frame}
    \frametitle{预期结果}
    \textbf{直接将 AD 引入变密度方法的效果}
    \begin{itemize}
        \item[•]\textbf{预期结果}:直接将 AD 技术引入变密度方法的效果可能不好,可能无法超越传统的变密度方法。
        \vspace{0.3cm}
        \item[•]\textbf{原因分析}:传统变密度方法经过多年的发展和优化,已经在很多方面达到了较好的效果,直接引入 AD 可能无法立即表现出优势。
    \end{itemize}
\end{frame}
%--------------------------------------------------------------------------------------

%--------------------------------------------------------------------------------------
\begin{frame}
    \frametitle{预期结果}
    \textbf{额外引入正则化项后的效果}
    \begin{itemize}
        \item[•]\textbf{预期结果}:在引入正则化项之后,应用 AD 的变密度方法可能比传统的变密度方法效果更好。
        \vspace{0.3cm}
        \item[•]\textbf{原因分析}:正则化项可以改善设计变量的分布,使得 AD 的优势在处理复杂问题和提高计算效率方面更加明显。
    \end{itemize}
\end{frame}
%--------------------------------------------------------------------------------------

%--------------------------------------------------------------------------------------
\begin{frame}
    \frametitle{贡献}
    \textbf{引入 AD 技术对拓扑优化研究的贡献}
    \begin{itemize}
        \item[•]目前将 AD 引入拓扑优化的研究工作比较少,特别是将其与传统方法进行比较的工作更少。本研究通过对比分析传统变密度方法和引入 AD 的变密度方法,提供了宝贵的实验数据和理论支持。
        \vspace{0.3cm}
        \item[•]有利于后续应对 AD 技术在非线性问题和多物理场问题中的应用挑战。
    \end{itemize}
\end{frame}
%--------------------------------------------------------------------------------------

%--------------------------------------------------------------------------------------
\begin{frame}
    \frametitle{贡献}
    \textbf{基于 FEALPy 的实现对拓扑优化研究的贡献}
    \begin{itemize}
        \item[•]基于 FEALPy 实现传统的变密度方法和引入 AD 的变密度方法,为后续拓扑优化研究工作的进一步展开提供了坚实的基础。
        \vspace{0.3cm}
        \item[•]FEALPy 的模块化设计和多计算内核支持,使得研究成果具有较高的可重用性和扩展性,利于后续研究人员在此基础上进行优化和创新。
    \end{itemize}
\end{frame}
%--------------------------------------------------------------------------------------


\end{document}
